%% This is file `elsarticle-template-1-num.tex',
%%
%% Copyright 2009 Elsevier Ltd
%%
%% This file is part of the 'Elsarticle Bundle'.
%% ---------------------------------------------
%%
%% It may be distributed under the conditions of the LaTeX Project Public
%% License, either version 1.2 of this license or (at your option) any
%% later version.  The latest version of this license is in
%%    http://www.latex-project.org/lppl.txt
%% and version 1.2 or later is part of all distributions of LaTeX
%% version 1999/12/01 or later.
%%
%% The list of all files belonging to the 'Elsarticle Bundle' is
%% given in the file `manifest.txt'.
%%
%% Template article for Elsevier's document class `elsarticle'
%% with numbered style bibliographic references
%%
%% $Id: elsarticle-template-1-num.tex 149 2009-10-08 05:01:15Z rishi $
%% $URL: http://lenova.river-valley.com/svn/elsbst/trunk/elsarticle-template-1-num.tex $
%%
\documentclass[preprint,12pt]{elsarticle}

%% Use the option review to obtain double line spacing
%% \documentclass[preprint,review,12pt]{elsarticle}

%% Use the options 1p,twocolumn; 3p; 3p,twocolumn; 5p; or 5p,twocolumn
%% for a journal layout:
%% \documentclass[final,1p,times]{elsarticle}
%% \documentclass[final,1p,times,twocolumn]{elsarticle}
%% \documentclass[final,3p,times]{elsarticle}
%% \documentclass[final,3p,times,twocolumn]{elsarticle}
%% \documentclass[final,5p,times]{elsarticle}
%% \documentclass[final,5p,times,twocolumn]{elsarticle}

%% if you use PostScript figures in your article
%% use the graphics package for simple commands
%% \usepackage{graphics}
%% or use the graphicx package for more complicated commands
%% \usepackage{graphicx}
%% or use the epsfig package if you prefer to use the old commands
%% \usepackage{epsfig}

%% The amssymb package provides various useful mathematical symbols
\usepackage{amssymb}
%% The amsthm package provides extended theorem environments
%% \usepackage{amsthm}

%% The lineno packages adds line numbers. Start line numbering with
%% \begin{linenumbers}, end it with \end{linenumbers}. Or switch it on
%% for the whole article with \linenumbers after \end{frontmatter}.
\usepackage{lineno}

\usepackage{hyperref}
\usepackage{quoting}



%% natbib.sty is loaded by default. However, natbib options can be
%% provided with \biboptions{...} command. Following options are
%% valid:

%%   round  -  round parentheses are used (default)
%%   square -  square brackets are used   [option]
%%   curly  -  curly braces are used      {option}
%%   angle  -  angle brackets are used    <option>
%%   semicolon  -  multiple citations separated by semi-colon
%%   colon  - same as semicolon, an earlier confusion
%%   comma  -  separated by comma
%%   numbers-  selects numerical citations
%%   super  -  numerical citations as superscripts
%%   sort   -  sorts multiple citations according to order in ref. list
%%   sort&compress   -  like sort, but also compresses numerical citations
%%   compress - compresses without sorting
%%
%% \biboptions{comma,round}

% \biboptions{}

\begin{document}

\begin{frontmatter}

%% Title, authors and addresses

%% use the tnoteref command within \title for footnotes;
%% use the tnotetext command for the associated footnote;
%% use the fnref command within \author or \address for footnotes;
%% use the fntext command for the associated footnote;
%% use the corref command within \author for corresponding author footnotes;
%% use the cortext command for the associated footnote;
%% use the ead command for the email address,
%% and the form \ead[url] for the home page:
%%
%% \title{Title\tnoteref{label1}}
%% \tnotetext[label1]{}
%% \author{Name\corref{cor1}\fnref{label2}}
%% \ead{email address}
%% \ead[url]{home page}
%% \fntext[label2]{}
%% \cortext[cor1]{}
%% \address{Address\fnref{label3}}
%% \fntext[label3]{}

\title{Confronto implementazioni Metodo di Cholesky\\
\large{Attraverso l'utilizzo di ambienti differenti: Matlab, C++ e R}}

%% use optional labels to link authors explicitly to addresses:
%% \author[label1,label2]{<author name>}
%% \address[label1]{<address>}
%% \address[label2]{<address>}

\author{Lorenzo Rovida, Nicolò Ripamonti\\
	\small{ \href{mailto:l.rovida1@campus.unimib.it}{l.rovida1@campus.unimib.it},  \href{mailto:n.ripamonti@campus.unimib.it}{n.ripamonti@campus.unimib.it}\\817151, 816171}}

\address{Dipartimento di Informatica, Sistemi e Comunicazione, Universitá degli Studi di Milano-Bicocca, Milano, Italia}

\begin{abstract}
Lo scopo di questo progetto è di studiare l'implementazione in ambienti di programmazione open source del metodo di \textbf{Choleski} per la risoluzione sistemi lineari per matrici sparse, simmetriche e definite positive e successivamente confrontarli con l’implementazione nell'ambiente di programmazione di Matlab.
	
Immaginiamo che un'azienda abbia la necessità di munirsi di un ambiente di programmazione per risolvere sistemi lineari con matrici sparse e definite positive di grandi dimensioni utilizzando il metodo di Choleski. 
	
L'alternativa è tra software proprietario (Matlab) oppure open source e anche tra Windows oppure Linux.
	
La richiesta è quindi quella di confrontare, in ambiente Linux e Windows, e sulla stessa macchina, l'ambiente di programmazione Matlab con una libreria open source. 
	
In altre parole bisogna porsi le seguenti domande:
\begin{enumerate}
		\item \`E meglio affidarsi alla sicurezza di Matlab pagando oppure vale la pena di avventurarsi nel mondo open source? 
		\item \`E meglio lavorare in ambiente Linux oppure in ambiente Windows?
\end{enumerate}
\end{abstract}
	
\begin{keyword}
Science \sep Publication \sep Complicated

\end{keyword}

\end{frontmatter}

%%
%% Start line numbering here if you want
%%

%% main text
\section{Introduzione} 

Lo scopo di questo progetto è quello di valutare l’implementazione, in ambienti di programmazione differenti, del metodo di Cholesky per la risoluzione di sistemi lineari per matrici sparse. La scelta degli ambienti, da parte del team, si è focalizzata su:

\begin{center}
	\textbf{Matlab} \quad - \quad \textbf{C++} \quad - \quad \textbf{R}
\end{center}

Le matrici che sono state considerate per la realizzazione del progetto sono quelle del gruppo \textit{SuiteSparse Matrix Collection} che colleziona matrici sparse derivanti da applicazioni di problemi reali (ingegneria strutturale, fluidodinamica, elettromagnetismo, termodinamica, computer graphics/vision, network e grafi).

In particolare le matrici simmetriche e definite positive che sono state analizzate sono le seguenti:
\begin{center}
	\begin{itemize}
		\item \textbf{Flan\_1565}
		\item \textbf{StocF-1465}
		\item \textbf{cfd2}
		\item \textbf{cfd1}
		\item \textbf{G3\_circuit}
		\item \textbf{parabolic\_fem}
		\item \textbf{apache2}
		\item \textbf{shallow\_water1}
		\item \textbf{ex15}
	\end{itemize}

\end{center}

Un sistema lineare si può rappresentare in forma matriciale come $ A \cdot x = b $
dove:
\begin{itemize}
	\item A è la matrice dei coefficienti del sistema ed è rappresentata da una delle matrici del gruppo \textit{SuiteSparse Matrix Collection} elencate precedentemente.
	\item x è il vettore colonna delle incognite.
	\item b è il vettore colonna dei termini noti; b è scelto in modo che la soluzione esatta sia il vettore $ xe = [1 1 1 \dots 1 1] $ avente tutte le componenti uguali a 1, tale che $ b = A \cdot xe $ .
\end{itemize}

\newpage

Il confronto dei vari ambienti di programmazione sui sistemi operativi di Windows e Linux deve avvenire in termini di:
\begin{description}
	\item[Tempo:] Calcolo dei tempi di esecuzione del metodo di Cholesky;
	\item[Accuratezza:] Calcolo dell'errore assoluto tra la soluzione esatta e la soluzione del sistema lineare;
	\item[Impiego della memoria:] Quantità di memoria utilizzata per il calcolo del sistema lineare con il metodo di Cholesky.
\end{description}

\subsection{Matrici sparse}
Le matrici sparse sono una classe di matrici con la caratteristica di contenere un significativo numero di elementi uguali a zero. Spesso il numero di elementi diversi da zero su ogni riga è un numero piccolo (per esempio del- l’ordine di $10^1$) indipendente dalla dimensione della matrice, che può essere anche dell’ordine di $10^8$.

Le matrici sparse si possono memorizzare in modo compatto, tenendo solo conto degli elementi diversi da zero; per esempio, è sufficiente per ogni elemento diverso da zero memorizzare solo la sua posizione $i j$ e il suo valore $a_{ij}$, ignorando gli elementi uguali a zero. Ciò consente di ridurre il tempo di calcolo eliminando operazioni su elementi nulli.

\section{Descrizione delle librerie}

\subsection{Matlab}
\begin{description}
	\item[Versione:] 9.7.0
	\item[Referenze:] \href{https://it.mathworks.com/products/matlab.html}{Sito} - \href{https://www.mathworks.com/help/matlab/}{Documentazione}
\end{description}

Matlab (abbreviazione di Matrix Laboratory) è un ambiente per il calcolo numerico e l'analisi statistica scritto in C, che comprende anche l'omonimo linguaggio di programmazione creato dalla MathWorks. MATLAB consente di manipolare matrici, visualizzare funzioni e dati, implementare algoritmi, creare interfacce utente, e interfacciarsi con altri programmi. Matlab è usato da milioni di persone nell'industria e nelle università per via dei suoi numerosi strumenti a supporto dei più disparati campi di studio applicati e funziona su diversi sistemi operativi, tra cui Windows, Mac OS, GNU/Linux e Unix.

\subsection{Eigen C++}
\begin{description}
	\item[Versione:] 3.7.7, Novembre 2019
	\item[Referenze:] \href{http://eigen.tuxfamily.org/index.php?title=Main_Page}{Sito} - \href{http://eigen.tuxfamily.org/dox/}{Documentazione}
\end{description}

\textit{Eigen} è una libreria di modelli C++ per l'algebra lineare, ovvero che supporta il calcolo e la risoluzione di matrici, vettori, solutori numerici e algoritmi correlati.

Supporta matrici di qualsiasi dimensione, dalle piccole matrici di dimensioni fisse alle matrici dense arbitrariamente grandi, fino alle matrici sparse e il suo ecosistema offre molte funzionalità specializzate come l'ottimizzazione non lineare, le funzioni di matrice, un solutore polinomiale e molto altro.

\textit{"L'implementazione di un algoritmo su Eigen è come copiare semplicemente lo pseudocodice."}

Eigen ha un buon supporto per il compilatore mentre si esegue una suite di test per garantire affidabilità e aggirare eventuali bug del compilatore. Eigen è anche standard C ++ 98 e mantiene tempi di compilazione molto ragionevoli.

Eigen è un software open source concesso in licenza con Mozilla Public License 2.0 dalla versione 3.1.1.

In particolare di Eigen sono state utilizzate le funzioni:
\begin{description}
	\item[\verb!SimplicialLDLT< >] Questa classe fornisce fattorizzazioni $LDL^ T$ Cholesky di matrici sparse che sono definite positive. La fattorizzazione consente di risolvere $AX = B$ dove $X$ e $B$ possono essere densi o sparsi. Al fine di ridurre il riempimento, viene applicata una permutazione simmetrica $P$ prima della fattorizzazione in modo tale che la matrice fattorizzata sia $PAP ^ {-1}$;
	\item[\verb!solve()] Funzione utilizzata per trovare la soluzione di x.
\end{description}


\subsection{Matrix R}
\begin{description}
	\item[Versione:] 1.2-18, Novembre 2018
	\item[Referenze:] \href{http://matrix.r-forge.r-project.org}{Sito} - \href{https://www.rdocumentation.org/packages/Matrix}{Documentazione}
\end{description}

Il pacchetto di R \textit{Matrix} fornisce un set di classi per matrici dense e sparse che estendono le classi base delle matrici già presenti in R. I metodi per un'ampia gamma di funzioni e operatori applicati agli oggetti di queste classi forniscono un accesso efficiente alla soluzione di sistemi lineari, matrici dense e matrici sparse.

Una caratteristica notevole del pacchetto è che ogni volta che una matrice viene fattorizzata, la fattorizzazione viene memorizzata come parte della matrice originale in modo che ulteriori operazioni sulla matrice possano riutilizzare questa fattorizzazione.

In particolare del pacchetto Matrix, sono state usate due funzioni:

\begin{description}
	\item[\verb!readMM()] Funzione utilizzata per leggere un file di tipo MatrixMarket. All'interno della funzione basta inserire il percorso del file;
	\item[\verb!chol()] Funzione che calcola la fattorizzazione di Choleski di una matrice quadrata simmetrica e definita positiva. Gli argomenti che si possono inserire all'interno della funzione sono:
	\begin{description}
		\item[x] Una matrice quadrata (sparsa o densa); se x non è definito positivo, viene segnalato un errore. per la nostra implementazione abbiamo utilizzato come argomento della funzione solamente la matrice;
		\item[pivot] Valore logico che indica se si deve usare il pivot. Attualmente, questo non viene utilizzato per matrici dense.
		\item[cache] Valore logico che indica se il risultato deve essere memorizzato nella cache; si noti che questo argomento è sperimentale e disponibile solo per alcune matrici sparse.
	\end{description}
\end{description}


\section{The Second Section}
\label{S:2}

Reference to Section \ref{S:1}. Etiam congue sollicitudin diam non porttitor. Etiam turpis nulla, auctor a pretium non, luctus quis ipsum. Fusce pretium gravida libero non accumsan. Donec eget augue ut nulla placerat hendrerit ac ut mi. Phasellus euismod ornare mollis. Proin tempus fringilla ultricies. Donec pretium feugiat libero quis convallis. Nam interdum ante sed magna congue eu semper tellus sagittis. Curabitur eu augue elit.

Aenean eleifend purus et massa consequat facilisis. Etiam volutpat placerat dignissim. Ut nec nibh nulla. Aliquam erat volutpat. Nam at massa velit, eu malesuada augue. Maecenas sit amet nunc mauris. Maecenas eu ligula quis turpis molestie elementum nec at est. Sed adipiscing neque ac sapien viverra sit amet vestibulum arcu rhoncus.

Vivamus pharetra nibh in orci euismod congue. Pellentesque habitant morbi tristique senectus et netus et malesuada fames ac turpis egestas. Quisque lacus diam, congue vel laoreet id, iaculis eu sapien. In id risus ac leo pellentesque pellentesque et in dui. Etiam tincidunt quam ut ante vestibulum ultricies. Nam at rutrum lectus. Aenean non justo tortor, nec mattis justo. Aliquam erat volutpat. Nullam ac viverra augue. In tempus venenatis nibh quis semper. Maecenas ac nisl eu ligula dictum lobortis. Sed lacus ante, tempor eu dictum eu, accumsan in velit. Integer accumsan convallis porttitor. Maecenas pretium tincidunt metus sit amet gravida. Maecenas pretium blandit felis, ac interdum ante semper sed.

In auctor ultrices elit, vel feugiat ligula aliquam sed. Curabitur aliquam elit sed dui rhoncus consectetur. Cras elit ipsum, lobortis a tempor at, viverra vitae mi. Cras sed urna sed eros bibendum faucibus. Morbi vel leo orci, vel faucibus orci. Vivamus urna nisl, sodales vitae posuere in, tempus vel tellus. Donec magna est, luctus non commodo sit amet, placerat et enim.

%% The Appendices part is started with the command \appendix;
%% appendix sections are then done as normal sections
%% \appendix

%% \section{}
%% \label{}

%% References
%%
%% Following citation commands can be used in the body text:
%% Usage of \cite is as follows:
%%   \cite{key}          ==>>  [#]
%%   \cite[chap. 2]{key} ==>>  [#, chap. 2]
%%   \citet{key}         ==>>  Author [#]

%% References with bibTeX database:

\bibliographystyle{model1-num-names}
\bibliography{sample.bib}

%% Authors are advised to submit their bibtex database files. They are
%% requested to list a bibtex style file in the manuscript if they do
%% not want to use model1-num-names.bst.

%% References without bibTeX database:

% \begin{thebibliography}{00}

%% \bibitem must have the following form:
%%   \bibitem{key}...
%%

% \bibitem{}

% \end{thebibliography}


\end{document}

%%
%% End of file `elsarticle-template-1-num.tex'.
