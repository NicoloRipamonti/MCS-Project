\begin{abstract}
Lo scopo di questo progetto è di studiare l'implementazione in ambienti di programmazione open source del metodo di \textbf{Choleski} per la risoluzione sistemi lineari per matrici sparse, simmetriche e definite positive e successivamente confrontarli con l’implementazione nell'ambiente di programmazione di MatLab.

Immaginiamo che un'azienda abbia la necessità di munirsi di un ambiente di programmazione per risolvere sistemi lineari con matrici sparse e definite positive di grandi dimensioni utilizzando il metodo di Choleski. 

L'alternativa è tra software proprietario (MatLab) oppure open source e anche tra Windows oppure Linux.

La richiesta è quindi quella di confrontare, in ambiente Linux e Windows, e sulla stessa macchina, l'ambiente di programmazione MatLab con una libreria open source. 

In altre parole bisogna porsi le seguenti domande:
\begin{enumerate}
\item \`E meglio affidarsi alla sicurezza di MatLab pagando oppure vale la pena di avventurarsi nel mondo open source? 
\item \`E meglio lavorare in ambiente Linux oppure in ambiente Windows?
\begin{quoting}
\end{abstract}
\addtocounter{page}{-1}