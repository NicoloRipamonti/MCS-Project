\begin{abstract}
Lo scopo di questo progetto è di studiare l'implementazione in ambienti di programmazione open source del metodo di \textbf{Choleski} per la risoluzione sistemi lineari per matrici sparse, simmetriche e definite positive e successivamente confrontarli con l’implementazione nell'ambiente di programmazione di MATLAB.

Immaginiamo che la vostra azienda abbia la necessità di munirsi di un ambiente di programmazione per risolvere con il metodo di Choleski sistemi lineari con matrici sparse e definite positive di grandi dimensioni. 

L'alternativa è tra software proprietario (MATLAB) oppure open source e anche tra Windows oppure Linux.

La richiesta è quindi quella di confrontare, in ambiente Linux e Windows, e sulla stessa macchina, l'ambiente di programmazione MATLAB con una libreria open source a nostra scelta. 

Il confronto deve avvenire in termini di:
\begin{description}
\item[Tempo:] Calcolo dei tempi di esecuzione del metodo di Cholesky;
\item[Accuratezza:] Calcolo dell'errore assoluto tra la soluzione esatta e la soluzione del sistema lineare;
\item[Impiego della memoria:] Quantità di memoria utilizzata per il calcolo del sistema lineare con il metodo di Cholesky.
\end{description}
In altre parole bisogna porsi le seguenti domande:
\begin{enumerate}
\item \`E meglio affidarsi alla sicurezza di MATLAB pagando oppure vale la pena di avventurarsi nel mondo open source? 
\item \`E meglio lavorare in ambiente Linux oppure in ambiente Windows?
\begin{quoting}
\end{abstract}
\addtocounter{page}{-1}