\section{10-Fold Cross Validation}
\label{10fcv}
\subsubsection*{Che cos'è?}


\subsubsection*{Implementazione}

All'interno dello script R, basta aggiungere un oggetto di tipo \textit{trainControl} alla funzione \textit{train()}.

Un oggetto \textit{trainControl} è cosi definito:

\begin{lstlisting}[caption=Training con train control]
model = train(targetColumn ~ ., 
	data = trainset, 
	method = 'svmRadial',
	metric = "ROC", 
	trControl = control)
\end{lstlisting}

Utilizzando questa tecnica, abbiamo addestrato i diversi modelli (ovvero: Decision Tree, SVM e Neural Network).


\section{Stima delle misure di performance}



\subsection{Matrice di confusione}
La matrice di confusione è una rappresentazione dell'accuratezza di classificazione statistica.











