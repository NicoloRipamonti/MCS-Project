\section{MathLab} 

\section{C++}
\subsection{Eigen}
\textit{Eigen} è una libreria di modelli C++ per l'algebra lineare, ovvero che supporta il calcolo e la risoluzione di matrici, vettori, solutori numerici e algoritmi correlati.

Supporta matrici di qualsiasi dimensione, dalle piccole matrici di dimensioni fisse alle matrici dense arbitrariamente grandi, fino alle matrici sparse e il suo ecosistema offre molte funzionalità specializzate come l'ottimizzazione non lineare, le funzioni di matrice, un solutore polinomiale e molto altro.

In Eigen, le matrici a dimensione fissa sono completamente ottimizzate: l'allocazione dinamica della memoria viene evitata e i loop vengono srotolati quando serve, mentre per le matrici di grandi dimensioni, viene prestata particolare attenzione alla compatibilità con la cache.

Gli algoritmi del pacchetto open source in questione sono accuratamente selezionati per l'affidabilità. I compromessi di affidabilità sono chiaramente documentati e sono disponibili decomposizioni estremamente sicure.

\textit{"L'implementazione di un algoritmo su Eigen è come copiare semplicemente lo pseudocodice."}

Eigen ha un buon supporto per il compilatore mentre si esegue una suite di test per garantire affidabilità e aggirare eventuali bug del compilatore. Eigen è anche standard C ++ 98 e mantiene tempi di compilazione molto ragionevoli.

Eigen è un software open source concesso in licenza con Mozilla Public License 2.0 dalla versione 3.1.1. 

Attualmente Eigen ha un'ottima \href{http://eigen.tuxfamily.org/dox/}{documentazione} e tale libreria viene tuttora mantenuta alla versione 3.7.7, rilasciata nel Novembre del 2018.

\section{R}
\subsection{Matrix}
Il pacchetto di R \textit{Matrix} fornisce un set di classi per matrici dense e sparse che estendono le classi base delle matrici già presenti in R. I metodi per un'ampia gamma di funzioni e operatori applicati agli oggetti di queste classi forniscono un accesso efficiente alla soluzione di sistemi lineari, matrici dense e matrici sparse.

Una caratteristica notevole del pacchetto è che ogni volta che una matrice viene fattorizzata, la fattorizzazione viene memorizzata come parte della matrice originale in modo che ulteriori operazioni sulla matrice possano riutilizzare questa fattorizzazione.

In particolare del pacchetto Matrix, sono state usate due funzioni:

\begin{description}
\item[\verb!readMM()] Funzione utilizzata per leggere un file di tipo MatrixMarket. All'interno della funzione basta inserire il percorso del file;
\item[\verb!chol()] Funzione che calcola la fattorizzazione di Choleski di una matrice quadrata simmetrica e definita positiva. Gli argomenti che si possono inserire all'interno della funzione sono:
\begin{description}
\item[x] Una matrice quadrata (sparsa o densa); se x non è definito positivo, viene segnalato un errore. per la nostra implementazione abbiamo utilizzato come argomento della funzione solamente la matrice;

\item[pivot] Valore logico che indica se si deve usare il pivot. Attualmente, questo non viene utilizzato per matrici dense.

\item[cache] Valore logico che indica se il risultato deve essere memorizzato nella cache; si noti che questo argomento è sperimentale e disponibile solo per alcune matrici sparse.
\end{description}
\end{description}
